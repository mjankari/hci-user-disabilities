\section{Conclusion}

In conclusion, this paper emphasized the importance of designing interfaces that
cater to users with disabilities. Following the POUR principles (Perceivable,
Operable, Understandable, and Robust) and incorporating design considerations
for specific needs creates inclusive and user-friendly experiences for a broader
audience. User-centered design with active participation from users with
disabilities throughout the process is essential. User testing methods like
assistive technology testing, keyboard navigation testing, and screen reader
testing help identify and address usability issues. While this paper focused on
core principles, future trends in accessible HCI technologies include AI-powered
tools for automatic alt text generation and real-time captioning, along with
advancements in eye-tracking, brain-computer interfaces, and gesture recognition
that could provide new interaction methods. As AR and VR technologies evolve,
ensuring accessibility will be crucial for their widespread adoption. By
embracing these trends and remaining committed to user-centered design, we can
continuously improve the accessibility of HCI and empower everyone to benefit
from technology.
