\section{Understanding Disabilities}

Human-computer interaction (HCI) needs to be inclusive of a wide range of
disabilities. This section explores some common types of disabilities and how
they can hinder interaction with traditional interfaces.

People with visual impairments, including blindness, low vision, and color
blindness, may struggle to see text, icons, or images clearly. Interfaces with
small fonts, poor color contrast, or a heavy reliance on visual cues can be
difficult or impossible for them to navigate.

Deaf or hard-of-hearing users cannot perceive audio cues like alerts,
notifications, or instructional videos. Interfaces that solely rely on audio
feedback create significant barriers for them.

Individuals with motor impairments, such as limited dexterity or tremors, may
struggle to use a mouse, keyboard, or touchscreen effectively. Interfaces
requiring precise movements or rapid keystrokes can be challenging for them.

The term ``cognitive disabilities'' encompasses a broad category of conditions
including learning disabilities, intellectual disabilities, and ADHD. Users with
cognitive disabilities may have difficulty understanding complex instructions,
remembering multi-step processes, or focusing on cluttered interfaces.

To address these challenges, accessibility standards and guidelines have been
established. A prominent example is the Web Content Accessibility Guidelines
(WCAG) set forth by the World Wide Web Consortium (W3C). WCAG outlines specific
criteria for web content to be perceivable, operable, understandable, and robust
(POUR principles) for users with disabilities. These guidelines provide a
framework for creating interfaces that are usable by everyone, regardless of
ability.
