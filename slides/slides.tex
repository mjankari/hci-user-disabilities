\documentclass{beamer}

\title{The Design of Interfaces for Users with Disabilities}
\author{Mahdi Jankari}
\date{May 8, 2024}

\begin{document}
\maketitle

\begin{frame}{Outline}
\tableofcontents
\end{frame}

\section{Introduction}
\begin{frame}{Introduction}
    \begin{itemize}
        \item In today's world, technology plays an increasingly important role in our lives.
        \item Human-Computer Interaction (HCI) is a field dedicated to bridging the gap between humans and the machines they use.
        \item HCI specialists focus on designing and evaluating computer systems, paying particular attention to how users interact with them.
        \item Their goal is to create technology that is not just functional, but also intuitive, efficient, and enjoyable to use.
        \item This involves a deep understanding of human capabilities and limitations, encompassing how we perceive information, learn new systems, and interact with the world around us.
    \end{itemize}
\end{frame}

\section{The Importance of HCI}
\begin{frame}{The Importance of HCI}
    \begin{itemize}
        \item Everyone deserves to benefit from technology.
        \item People with disabilities face barriers when using traditional interfaces
        \item Accessible interfaces can help to remove these barriers
    \end{itemize}
\end{frame}

\section{POUR Principles}
\begin{frame}{POUR Principles}
    \begin{itemize}
        \item \alert{Perceivable:} Information must be presented in a way that can be perceived by users with different disabilities.
        \item \alert{Operable:} Interfaces must be operable by users with different disabilities.
        \item \alert{Understandable:} Content must be understandable by users with different cognitive abilities.
        \item \alert{Robust:} Content must be robust and compatible with a wide range of assistive technologies.
    \end{itemize}
\end{frame}

\section{Design Considerations for Different Disabilities}
\begin{frame}{Design Considerations for Different Disabilities}
    \begin{itemize}
        \item \alert{Visual impairments:} Ensure adequate color contrast, provide alt text for images, offer options for adjusting font size
        \item \alert{Hearing impairments:} Provide captions for videos, offer transcripts of audio content, use visual cues alongside audio alerts
        \item \alert{Motor impairments:} Use large touch targets, support keyboard navigation, integrate voice control options
        \item \alert{Cognitive disabilities:} Use clear and concise language, break down complex information,
    \end{itemize}
\end{frame}

\section{User-Centered Design for Accessibility}
\begin{frame}{User-Centered Design for Accessibility}
    \begin{itemize}
        \item Involve users with disabilities in the design process
        \item User testing is essential to identify and address usability issues
        \item User collaboration fosters empathy and a deeper understanding of the needs of users with disabilities
    \end{itemize}
\end{frame}

\section{The Expanding Benefits of Accessible Design}
\begin{frame}{The Expanding Benefits of Accessible Design}
    \begin{itemize}
        \item Caters to a wider user base
        \item Ethical and social imperative of inclusion
        \item The business case for accessibility
    \end{itemize}
\end{frame}
\end{document}
